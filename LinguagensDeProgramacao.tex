\documentclass[conference]{IEEEtran}
\IEEEoverridecommandlockouts
% The preceding line is only needed to identify funding in the first footnote. If that is unneeded, please comment it out.
\usepackage{enumerate}% http://ctan.org/pkg/enumerate

\usepackage{graphicx,url}

\usepackage[brazil]{babel}   
%\usepackage[latin1]{inputenc}  
\usepackage[utf8]{inputenc}  
% UTF-8 encoding is recommended by ShareLaTex
\usepackage{cite}
\usepackage{amsmath,amssymb,amsfonts}
\usepackage{listings}
\usepackage{color}
\usepackage{algorithmic}
\usepackage{graphicx}
\usepackage{textcomp}
\usepackage{xcolor}
\usepackage{comment}
\definecolor{codegreen}{rgb}{0,0.6,0}
\definecolor{codegray}{rgb}{0.5,0.5,0.5}
\definecolor{codepurple}{rgb}{0.58,0,0.82}
\definecolor{backcolour}{rgb}{0.95,0.95,0.92}
\renewcommand{\lstlistingname}{Código Fonte}% Listing -> Algorithm
\lstdefinestyle{mystyle}{
  backgroundcolor=\color{backcolour},   commentstyle=\color{codegreen},
  keywordstyle=\color{magenta},
  numberstyle=\tiny\color{codegray},
  stringstyle=\color{codepurple},
  basicstyle=\footnotesize,
  breakatwhitespace=false,         
  breaklines=true,                 
  captionpos=b,                    
  keepspaces=true,                 
  numbers=left,                    
  numbersep=5pt,                  
  showspaces=false,                
  showstringspaces=false,
  showtabs=false,                  
  tabsize=2
}

%"mystyle" code listing set
\lstset{style=mystyle}
\def\BibTeX{{\rm B\kern-.05em{\sc i\kern-.025em b}\kern-.08em
    T\kern-.1667em\lower.7ex\hbox{E}\kern-.125emX}}
\begin{document}

\title{Linguagens de programação: história e estrutura\\
}

\author{\IEEEauthorblockN{Patrick Araújo}
\IEEEauthorblockA{\textit{Acadêmico do curso de Ciência da Computação} \\
\textit{Universidade Federal do Tocantins}\\
Palmas - TO, Brasil \\
patrick.araujo@uft.edu.br}
}

\maketitle

\begin{abstract}
Diante da necessidade de elaborar um trabalho acadêmico para a disciplina de Linguagens de Computação, o presente artigo apresenta dez linguagens de programação com o objetivo de mostrar um pouco da sua estrutura e sintaxe além de oferecer uma breve introdução à sua história. A escolha das linguagens se baseia pela sua importância histórica e relevância nos dias atuais. Para cada linguagem será mostrado um tutorial de como instalar as ferramentas necessárias para executar os programas no sistema operacional \textit{Windows}.
\begin{comment}
Diante da necessidade de elaborar um trabalho acadêmico para a disciplina de Linguagens de Computação, o presente artigo apresenta dez linguagens de programação com o objetivo de mostrar um pouco da sua estrutura e sintaxe além de oferecer uma breve introdução à sua história. A escolha das linguagens se baseia pela sua importância histórica e relevância nos dias atuais. Para cada linguagem será mostrado um tutorial de como instalar as ferramentas necessárias para executar os programas no sistema operacional Windows.
\end{comment}
\end{abstract}

\begin{IEEEkeywords}
Linguagens de Programação, Evolução, História
\end{IEEEkeywords}

\section{Introdução}
Nos primórdios da computação, os programadores tinham que escrever programas inteiros em código binário. Mais especificamente, eles escreviam uma versão de alto nível do programa no papel. \textit{Plankalkül} foi a primeira linguagem alto nível a surgir, criada em 1942, na Alemanha nazista, pelo cientista Konrad Zuse, e permaneceu sob sigilo até ser amplamente divulgada em 1972. 

Em 1954, surge a \textit{FORTRAN} (\textit{IBM Mathematical Formula Translation System}), criada nos laboratórios da \textit{IBM} pelo cientista John Barckus, o compilador de \textit{FORTRAN} foi desenvolvido para o computador \textit{IBM} 704.  Essa foi a primeira linguagem de programação imperativa. Sua última versão se chama \textit{FORTRAN} 2008.

Surgida em 1972, a linguagem \textit{C} ainda continua popular ao redor do mundo. Ela foi criada por Dennis Ritchie nos laboratórios da \textit{AT\&T} para o desenvolvimento do sistema \textit{UNIX}. É uma linguagem de paradigma estruturado, imperativa e procedural. Graças à \textit{C}, outras linguagens surgiram como o \textit{C++} e o \textit{Java}.

A sigla \textit{COBOL} significa \textit{Common, Business Oriented Language}, ou Linguagem Comum Orientada para os Negócios. Ainda popular em sistemas bancários, ela é uma linguagem orientada para o processamento de banco de dados comerciais. Ela surge em 1959 como fruto das pesquisas de Gracee Hopper, que trabalhava no Departamento de Defesa americano. Possui paradigma procedural, imperativo e orientado a objetos.

A linguagem \textit{Java} surgiu em 1995 e foi criada por James Gosling, que trabalhava na empresa \textit{Sun Microsystems}, hoje \textit{Oracle}. Bastante usada no meio acadêmico para ensino de orientação a objetos, é umas das linguagens de programação mais populares atualmente. Umas das razões dessa popularidade se justifica pelo fato de ela ser multiplataforma, isso é possível porque o seu o código é compilado para \textit{bytecode} que, em seguida, é interpretado pela \textit{Java Virtual Machine}.

Criada por Guido van Rossum e lançada em 1991, a linguagem \textit{Python} se configura como umas das linguagens mais populares atualmente. A filosofia de design de \textit{Python} enfatiza a legibilidade de código que permite construir programas limpos e com poucas linhas. Possui o desenvolvimento aberto que é gerido pela \textit{Python Software Foudation}. Seu paradigma é de orientação a objetos.

Desenvolvida pela \textit{Microsoft}, em 2000, a \textit{C\#} faz parte da plataforma de desenvolvimento \textit{.NET}. Possui bastante recursos da linguagem \textit{Java}, \textit{C++} e \textit{Pascal} e é orientada a objetos. Também como a \textit{Java}, seu código é compilado para \textit{CIL} (\textit{Common Intermediate Language}) que, em seguida,  é interpretado pela máquina virtual \textit{Common Language Runtime} (\textit{CLR}).

A \textit{PHP} surgiu em 1995, orientada a objetos, é bastante usada para o desenvolvimento do lado servidor de aplicações \textit{web}. Foi criada por Rasmus Lerdorf.

Surgida em 1993 pelos desenvolvedores Roberto Ierusalimschy, Luiz Henrique de Figueiredo e Waldemar Celes,  a linguagem \textit{Lua} é muito utilizada em desenvolvimento de jogos. Ela foi criada nos laboratórios da PUC-RIO e é multiparadigma.

Nomeada em homenagem ao físico Blaise Pascal, \textit{Pascal} surgiu em 1970 criada por Niklaus Wirth. É uma linguagem procedural muito utilizada no ensino de programação.

\textit{Prolog} é uma linguagem declarativa que surgiu em 1972, criada por Alain Colmerauer e Robert Kowalski na Universidade de Marselha. Foi muito utilizada para soluções de problemas lógicos e inteligência artificial.

As linguagens de programação que serão tratadas nesse artigo são: \textit{FORTRAN}, \textit{C}, \textit{COBOL}, \textit{Java}, \textit{Python}, \textit{C\#}, \textit{PHP}, \textit{Lua}, \textit{Pascal} e \textit{Prolog}.

\begin{comment}
Nos primórdios da computação, os programadores tinham que escrever programas inteiros em código binário. Mais especificamente, eles escreviam uma versão de alto nível do programa no papel. Plankalkül foi a primeira linguagem alto nível a surgir, criada em 1942, na Alemanha nazista, pelo cientista Konrad Zuse, e permaneceu sob sigilo até ser amplamente divulgada em 1972. 

Em 1954, surge a FORTRAN (IBM Mathematical Formula Translation System), criada nos laboratórios da IBM pelo cientista John Barckus, o compilador de FORTRAN foi desenvolvido para o computador IBM 704.  Essa foi a primeira linguagem de programação imperativa. Sua última versão se chama Fortran 2008.

Surgida em 1972, a linguagem C ainda continua popular ao redor do mundo. Ela foi criada por Dennis Ritchie nos laboratórios da AT&T para o desenvolvimento do sistema UNIX. É uma linguagem de paradigma estruturado, imperativa e procedural. Graças à C, outras linguagens surgiram como o C++ e o Java.

A sigla COBOL significa Common, Business Oriented Language, ou Linguagem Comum Orientada para os Negócios. Ainda popular em sistemas bancários, ela é uma linguagem orientada para o processamento de banco de dados comerciais. Ela surge em 1959 como fruto das pesquisas de Gracee Hopper, que trabalhava no Departamento de Defesa americano. Possui paradigma procedural, imperativo e orientado a objetos.

A linguagem Java surgiu em 1995 e foi criada por James Gosling, que trabalhava na empresa Sun Microsystems, hoje Oracle. Bastante usada no meio acadêmico para ensino de orientação a objetos, é umas das linguagens de programação mais populares atualmente. Umas das razões dessa popularidade se justifica pelo fato de ela ser multiplataforma, isso é possível porque o seu o código é compilado para bytecode que, em seguida, é interpretado pela Java Virtual Machine.

Criada por Guido van Rossum e lançada em 1991, a linguagem Python se configura como umas das linguagens mais populares atualmente. A filosofia de design de Python enfatiza a legibilidade de código que permite construir programas limpos e com poucas linhas. Possui o desenvolvimento aberto que é gerido pela Pyhton Software Foudation. Seu paradigma é de orientação a objetos.

Desenvolvida pela Microsof, em 2000, a C# faz parte da plataforma de desenvolvimento .NET. Possui bastante recursos da linguagem Java, C++ e Pascal e é orientada a objetos. Também como a Java, seu código é compilado para CIL(Common Intermediate Language) que, em seguida,  é interpretado pela máquina virtual Common Language Runtime(CLR).

A PHP surgiu em 1995, orientada a objetos, é bastante usada para o desenvolvimento do lado servidor de aplicações web. Foi criada por Rasmus Lerdorf.

Surgida em 1993 pelos desenvolvedores Roberto Ierusalimschy, Luiz Henrique de Figueiredo e Waldemar Celes,  a linguagem Lua é muito utilizada em desenvolvimento de jogos. Ela foi criada nos laboratórios da PUC-RIO e é multiparadigma.

Nomeada em homenagem ao físico Blaise Pascal, Pascal surgiu em 1970 criada por Niklaus Wirth. É uma linguagem procedural muito utilizada no ensino de programação.

Prolog é uma linguagem declarativa que surgiu em 1972, criada por Alain Colmerauer e Robert Kowalski na Universidade de Marselha. Foi muito utilizada para soluções de problemas lógicos e inteligência artificial.

As linguagens de programação que serão tratadas nesse artigo são: FORTRAN, C, COBOL, JAVA, PYTHON, C#, PHP, LUA, PASCAL e PROLOG.
\end{comment}

\section{Instalando as ferramentas necessárias}

Para executar programas \textit{FORTRAN} e \textit{C} em ambientes Windows precisamos baixar primeiro o \textit{IDE Code::Blocks} . Para fazer isso vá até o site http://www.codeblocks.org e baixe o arquivo \textit{codeblocks-17.12mingw\_fortran-setup.exe}.
Faça a instalação do pacote e depois de instalado será possível criar algoritmos em \textit{FORTRAN} e em \textit{C}.

A execução de programas em \textit{COBOL} requer que baixemos o \textit{OpenCobol IDE}  disponível para \textit{Windows}. Para baixar o \textit{IDE} vá até o site http://opencobolide.readthedocs.io/en/latest/download.html e baixe o arquivo executável para \textit{Windows}.

Em \textit{Java} utilizaremos o \textit{IDE Netbeans}. Para baixar o \textit{IDE} vá até o site https://netbeans.org/downloads/. Também será necessário para executar algoritmos em Java que baixemos o \textit{Java Development Kit} (\textit{JDK}) no site https://www.oracle.com/technetwork/java/javase

No site https://www.python.org está disponível a \textit{Python Suite} para \textit{Windows}.

Com \textit{C\#} utilizaremos a \textit{IDE Visual Studio} disponível no site: https://visualstudio.microsoft.com/pt-br/.

Para executar algoritmos em \textit{PHP} utilizaremos o editor de texto \textit{Brackets}. Para baixa-lo vá até o site http://brackets.io. Precisaremos também do ambiente de desenvolvimento \textit{XAMPP} através do \textit{link} https://www.apachefriends.org/pt\_br/index.html é possível obtê-lo.

Na linguagem \textit{Lua} utilizaremos o \textit{IDE ZeroBrane} disponível para Windows através do site https://studio.zerobrane.com/download.html?not-this-time.

O processo com \textit{Pascal} requer baixar o \textit{IDE Dev-Pascal}. Para tanto vá até o site http://www.bloodshed.net/devpascal.html e baixe o executável para Windows.

Com \textit{Prolog} utilizaremos o \textit{IDE SWI Prolog}. Para baixá-lo vá até o site http://www.swi-prolog.org/download/stable e baixe a última versão do executável para Windows.

\section{Exemplos de impressão de "Hello Word"}

Abaixo iremos demonstrar como imprimir "Hello Word" nas diversas linguagens.

\begin{enumerate}[I]

\item \textit{FORTRAN}

\begin{lstlisting}[language=Fortran, caption=Exemplo em \textit{FORTRAN}]
program hello
    print *, "Hello World!"
end program hello
\end{lstlisting}


\item \textit{C}

\begin{lstlisting}[language=C, caption=Exemplo em \textit{C}]
#include <stdio.h>
#include <stdlib.h>

int main()
{
    printf("Hello world!\n");
    return 0;
}
\end{lstlisting}


\item \textit{COBOL}

\begin{lstlisting}[language=Cobol, caption=Exemplo em \textit{COBOL}]
IDENTIFICATION DIVISION.
PROGRAM-ID. HELLO-WORLD.
* simple hello world program
PROCEDURE DIVISION.
    DISPLAY 'Hello world!'.
STOP RUN.
\end{lstlisting}

\item \textit{Java}

\begin{lstlisting}[language=Java, caption=Exemplo em \textit{Java}]
/*
 * To change this license header, choose License Headers in Project Properties.
 * To change this template file, choose Tools | Templates
 * and open the template in the editor.
 */
package javaapplication2;

/**
 *
 * @author arauj
 */
public class JavaApplication2 {

    /**
     * @param args the command line arguments
     */
    public static void main(String[] args) {
        // TODO code application logic here
        System.out.println("Hello World!");
    }
    
}
\end{lstlisting}

\item \textit{Python}

\begin{lstlisting}[language=Python, caption=Exemplo em \textit{Python}]
print("Hello World!")
\end{lstlisting}

\item \textit{C\#}

\begin{lstlisting}[language=C, caption=Exemplo em \textit{C\#}]
// A Hello World! program in C#.
using System;
namespace HelloWorld
{
    class Hello 
    {
        static void Main() 
        {
            Console.WriteLine("Hello World!");

            // Keep the console window open in debug mode.
            Console.WriteLine("Press any key to exit.");
            Console.ReadKey();
        }
    }
}
\end{lstlisting}

\item \textit{PHP}

\begin{lstlisting}[language=PHP, caption=Exemplo em \textit{PHP}]
<? php
    echo "Hello World!";
?>
\end{lstlisting}

\item \textit{Lua}

\begin{lstlisting}[language=Python, caption=Exemplo em \textit{Lua}]
print("Hello World!")
\end{lstlisting}

\item \textit{Pascal}

\begin{lstlisting}[language=Pascal, caption=Exemplo em \textit{Pascal}]
program Hello;
begin
  writeln ('Hello, world.');
end.
\end{lstlisting}

\item \textit{Prolog}

\begin{lstlisting}[language=Prolog, caption=Exemplo em \textit{Prolog}]
?- write('Hello World!'), nl.
Hello world!
true.

?-
\end{lstlisting}

\end{enumerate}


\section*{Conclusão}

As diferenças entre as linguagens são evidentes tanto em questão de estrutura e como a sua história que justifica sua criação. Enquanto algumas linguagens possuem até 22 linhas, para fazer o mesmo processo uma linha seria necessária dependendo da linguagem. Algumas linguagens caíram no desuso popular, enquanto outras resistem ao tempo e se mantém populares junto a linguagens mais recentes. O futuro pode mudar essa realidade dependendo de novas tendências tecnológicas.

\begin{thebibliography}{00}
\bibitem{b1} Wikipédia, a enciclopédia livre, 'Plankalkül', 2018 [Online]. Available: https://pt.wikipedia.org/wiki/Plankalkül. [Accessed: 13- Aug- 2018].
\bibitem{b2} Wikipédia, a enciclopédia livre, 'Fortran', 2018 [Online]. Available: https://pt.wikipedia.org/wiki/Fortran. [Accessed: 13- Aug- 2018].
\bibitem{b3} Wikipédia, a enciclopédia livre, 'C', 2018 [Online]. Available: https://pt.wikipedia.org/wiki/C\_(linguagem\_de\_programação). [Accessed: 13- Aug- 2018].
\bibitem{b4} Wikipédia, a enciclopédia livre, 'COBOL', 2018 [Online]. Available: https://pt.wikipedia.org/w/index.php?title=COBOL&oldid=52669987. [Accessed: 13- Aug- 2018].
\bibitem{b5} Wikipédia, a enciclopédia livre, 'Java', 2018 [Online]. Available: https://pt.wikipedia.org/wiki/COBOL\_(linguagem\_de\_programação). [Accessed: 13- Aug- 2018].
\bibitem{b6} Wikipédia, a enciclopédia livre, 'Python', 2018 [Online]. Available: https://pt.wikipedia.org/w/index.php?title=Python&oldid=52882273. [Accessed: 13- Aug- 2018].
\bibitem{b7} Wikipédia, a enciclopédia livre, 'C\#', 2018 [Online]. Available: https://en.wikipedia.org/wiki/C\_Sharp\_(programming\_language). [Accessed: 13- Aug- 2018].
\bibitem{b8} Yuri. Pacievitch, InfoEscola, 'C\#', 2018 [Online]. Available: https://www.infoescola.com/informatica/c-sharp/. [Accessed: 13- Aug- 2018].
\bibitem{b9} Wikipédia, a enciclopédia livre, 'PHP', 2018 [Online]. Available: https://en.wikipedia.org/w/index.php?title=PHP&oldid=854762143. [Accessed: 13- Aug- 2018].
\bibitem{b10} Wikipédia, a enciclopédia livre, 'Lua', 2018 [Online]. Available: https://pt.wikipedia.org/wiki/Lua\_(linguagem\_de\_programação). [Accessed: 13- Aug- 2018].
\bibitem{b11} Wikipédia, a enciclopédia livre, 'Pascal', 2018 [Online]. Available: https://pt.wikipedia.org/wiki/Pascal\_(linguagem\_de\_programação). [Accessed: 13- Aug- 2018].
\bibitem{b12} Wikipédia, a enciclopédia livre, 'Prolog', 2018 [Online]. Available: https://pt.wikipedia.org/w/index.php?title=Prolog&oldid=49270619. [Accessed: 13- Aug- 2018].
\bibitem{b13} OpenCobolIDE, 'Download & Install — OpenCobolIDE 4.7.6 documentation', 2018 [Online]. Available: https://opencobolide.readthedocs.io/en/latest/download.html. [Accessed: 13- Aug- 2018].
\bibitem{b14} Code::Blocks, 2018 [Online]. Available: http://www.codeblocks.org. [Accessed: 13- Aug- 2018].
\bibitem{b15} NetBeans, 'Download o NetBeans IDE 8.2', 2018 [Online]. Available: https://netbeans.org/downloads/. [Accessed: 13- Aug- 2018].
\bibitem{b16} Oracle, 'Java SE | Oracle Technology Network | Oracle', 2018 [Online]. Available: https://www.oracle.com/technetwork/java/javase/overview/index.html. [Accessed: 13- Aug- 2018].
\bibitem{b17} Microsoft, 'Visual Studio IDE, Editor de Código, Team Services e Mobile Center', 2018 [Online]. Available: https://visualstudio.microsoft.com/pt-br/. [Accessed: 13- Aug- 2018].
\bibitem{b18} Brackets, 'Brackets - Um editor de textos moderno e de código aberto que entende web design.', 2018 [Online]. Available: http://brackets.io. [Accessed: 13- Aug- 2018].
\bibitem{b19} Apache Friends, 'XAMPP Installers and Downloads for Apache Friends', 2018 [Online]. Available: https://www.apachefriends.org/pt\_br/index.html. [Accessed: 13- Aug- 2018].
\bibitem{b20} ZeroBrane Studio, 'Download - ZeroBrane Studio - Lua IDE/editor/debugger for Windows, Mac OSX, and Linux', 2018 [Online]. Available: https://studio.zerobrane.com/download.html?not-this-time. [Accessed: 13- Aug- 2018].
\bibitem{b21} BloodshedSoftware, 'Bloodshed Software - Dev-Pascal', 2018 [Online]. Available: http://www.bloodshed.net/devpascal.html. [Accessed: 13- Aug- 2018].
\bibitem{b22} SWI-Prolog, 'SWI-Prolog downloads', 2018 [Online]. Available: http://www.swi-prolog.org/download/stable. [Accessed: 13- Aug- 2018].
\bibitem{b23} The First Programming Languages: Crash Course Computer Science #11. YouTube: 
CrashCourse, 2017.

\end{thebibliography}

\end{document}
